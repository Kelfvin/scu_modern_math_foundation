\section{非线性方程求根}

已知方程 $f(x) = 0$ ,在区间 $[a,b]$ 内有根,构造两个不同的收敛格式,证明收敛性并判断哪个格式收敛更快。

\subsection{迭代格式}

\subsubsection*{变换法}

通过变换(除法、开方等)将方程改写为 $x=\varphi(x)$ 

迭代格式: $x^{(k+1)} = \varphi(x^{(k)})$ 

全局收敛条件:1)在$[a,b]$内,$\varphi(x) \in [a,b]$;2)$\varphi$ 可导,且存在 $L<1$ 使得 $|\varphi'(x)|\leq L, \forall x \in [a,b]$

\subsubsection*{牛顿法}

\[
x_{k+1} = x_{k} - \frac{f(x)}{f'(x)}
\]

收敛条件:1)$f(a)f(b)<0$;2)$f'(x) \neq 0$,$\forall x \in [a,b]$;3)在$[a,b]$内,$f''(x)$ 不变号。则取初值 $x_0 \in [a,b]$ 且满足 $f(x_0)f''(x_0)>0$ ,则迭代收敛。

\subsection{收敛及收敛阶数}

设置$\alpha$为$f(\alpha)=0$在 $[a,b]$ 内的准确根(不用$x^*$是因为$\alpha$好写)。

局部收敛判定: 如果 $|\varphi'(\alpha)|<1$,且在 $\alpha$ 邻近连续,则初值够近就收敛。

收敛阶数:设 $x=\varphi(x)$的根具有连续$p$阶导数,且在根处 $\varphi'(\alpha)=\varphi''(\alpha)=...=\varphi^{(p-1)}(\alpha)=0$,$\varphi^{(p)}(\alpha) \neq 0$,则迭代格式的收敛阶数为 $p$ 。

% TODO: 重根的情况