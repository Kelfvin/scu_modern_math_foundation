\section{Hermite插值}

给定$n$ 个点的函数值,给定 $m$ 个点的导数值,求 $H(x)$ 和插值余项。

\subsection{基本解法:插值多项式+导数补偿}

1. 列出下列形式的插值多项式:
\[
H(x) =\underbrace{N_n(x)}_\text{n个点插值}  + 
\underbrace{
    \overbrace{(Ax^{m-1}+Bx^{m-2}\cdots+Z)}^{\text{m为给定导数点的个数}} \times
    \overbrace{(x-x_0)(x-x_1)\cdots(x-x_n)}^{x_i\text{为给定函数值的点}(i=0,1,\dots,n)}
}_\text{导数补偿项}
\]

2. 使用Newton插值公式求 $N_n(x)$。

3. 对 H(x) 求导数,并将导数条件代入,解出待定多项式的系数。

\subsubsection*{NewTon插值}

\begin{align*}
f[x_{i},x_{j}] &= \frac{f(x_j)-f(x_i)}{x_{j}-x_{i}} &&\text{一阶差商} \\
f[x_{0},x_{1},x_{2},\dots,x_{k}] &= \frac{f[x_{1},x_{2},\dots ,x_{k}]-f[x_{0},x_{1},\dots,x_{k-1}]}{x_{k}-x_{0}} &&\text{k阶差商}
\end{align*} 

对$x_0,x_1,x_2,x_3$按照从小到大的顺序排列

\begin{tabular}{|c|c|c|c|c|}
    \hline
    $x_k$ & $f(x)$ & 1阶差商 & 2阶差商 & 3阶差商 \\
    \hline
    $x_0$ & $f(x_0)$ & & & \\
    \hline
    $x_1$ & $f(x_1)$ & $f[x_0,x_1]$ & & \\
    \hline
    $x_2$ & $f(x_2)$ & $f[x_1,x_2]$ & $f[x_0,x_1,x_2]$ & \\
    \hline
    $x_3$ & $f(x_3)$ & $f[x_3,x_2]$ & $f[x_2,x_1,x_3]$ & $f[x_3,x_2,x_1,x_0]$ \\
    \hline
\end{tabular}

NewTon 插值公式:

$
   N_n(x) = f(x_0)
    + f[x_0, x_1](x - x_0)
    + \cdots
    + f[x_0, \dots, x_n](x - x_0)\cdots(x - x_{n-1})
$

$
    R_{n}(x) = f[x,x_{0},\dots,x_{n}]\omega(x)
$


\subsection{特殊情况:因式分解法(推荐使用)}

\subsubsection*{模板 1:函数值相同}
若已知$f(a)=c,\quad f(b)=c\quad (a\ne b)$(两个点相同),则

\[
\boxed{H(x)=c+(x-a)(x-b)\,Q(x)}.
\]

如果题目知道 $H(x)$ 的次数上界为 $n$,则 $Q(x)$ 的次数$m \le n-2$。

$Q(x)=Ax^{m}+Bx^{m-1}+\cdots+Zx^{0}$

随后把剩余条件代入,解 $Q(x)$ 的系数即可。

扩展:如果3个点值相同则增加1项然后$Q(x)$次数变为$n-3$,以此类推。

\subsubsection*{模板 2:同一点函数值和一阶导都给出(重零点)}
若已知$H(a)=c,\quad H'(a)=c',$则令$G(x)=H(x)-c-c'(x-a),$设
\[
\boxed{H(x)=c+c'(x-a)+(x-a)^2\,Q(x)}.
\]
同样若 $\deg H\le n$,则 $\deg Q\le n-2$。

\subsection{Hermite插值余项}:

\[
R(x) = \frac{f^{(n+m)}(\xi) }{(n+m)!} \prod_{i=1}^n (x - x_i) \prod_{x_j \in D}(x-x_j)
\]

其中 $D$ 是 $m$ 个给出导数值的点的集合。