\documentclass[12pt,a4paper,twocolumn]{ctexart}

% 控制页面边距
\usepackage[left=6mm,right=6mm,top=8mm,bottom=8mm]{geometry}

% ========= 引入常用宏包 ========= %

% 引入数学公式宏包
\usepackage{amsmath}

% 引入绘图宏包
\usepackage{tikz}
\usetikzlibrary{matrix,calc,positioning}

% 控制行间距
\usepackage{setspace}
\setstretch{0.92}  % 0.9~1.0 之间可调

% 压缩段落间距、标题压缩
\usepackage{titlesec}

\titlespacing\section{0pt}{4pt}{2pt}
\titlespacing\subsection{0pt}{3pt}{1pt}

\titlespacing\subsubsection{0pt}{3pt}{1pt}

\titleformat{\section}{\small\bfseries}{\thesection}{0.5em}{}
\titleformat{\subsection}{\small\bfseries}{\thesubsection}{0.4em}{}

% 数学公式压缩
\abovedisplayskip=4pt
\belowdisplayskip=4pt

% 列表压缩
\usepackage{enumitem}
\setlist{nosep,leftmargin=1.5em}  % 去除段间距 + 减少缩进

% 引入多列环境
\usepackage{multicol}

\usepackage{multirow}

% 去除首行缩进
\setlength{\parindent}{0pt}

\setlength{\columnsep}{10pt}        % 默认大概 10pt,可试 12pt
\setlength{\columnseprule}{0.2pt}   % 加一条细竖线(不想要就设 0pt)



\begin{document}


% 将正文默认字体改为 8pt
\renewcommand\normalsize{\fontsize{8}{9.2}\selectfont}
\normalsize

\section{Hermite插值}

给定$n$ 个点的函数值,给定 $m$ 个点的导数值,求 $H(x)$ 和插值余项。

\subsection{基本解法:插值多项式+导数补偿}

1. 列出下列形式的插值多项式:
\[
H(x) =\underbrace{N_n(x)}_\text{n个点插值}  + 
\underbrace{
    \overbrace{(Ax^{m-1}+Bx^{m-2}\cdots+Z)}^{\text{m为给定导数点的个数}} \times
    \overbrace{(x-x_0)(x-x_1)\cdots(x-x_n)}^{x_i\text{为给定函数值的点}(i=0,1,\dots,n)}
}_\text{导数补偿项}
\]

2. 使用Newton插值公式求 $N_n(x)$。

3. 对 H(x) 求导数,并将导数条件代入,解出待定多项式的系数。

\subsubsection*{NewTon插值}

\begin{align*}
f[x_{i},x_{j}] &= \frac{f(x_j)-f(x_i)}{x_{j}-x_{i}} &&\text{一阶差商} \\
f[x_{0},x_{1},x_{2},\dots,x_{k}] &= \frac{f[x_{1},x_{2},\dots ,x_{k}]-f[x_{0},x_{1},\dots,x_{k-1}]}{x_{k}-x_{0}} &&\text{k阶差商}
\end{align*} 

对$x_0,x_1,x_2,x_3$按照从小到大的顺序排列

\begin{tabular}{|c|c|c|c|c|}
    \hline
    $x_k$ & $f(x)$ & 1阶差商 & 2阶差商 & 3阶差商 \\
    \hline
    $x_0$ & $f(x_0)$ & & & \\
    \hline
    $x_1$ & $f(x_1)$ & $f[x_0,x_1]$ & & \\
    \hline
    $x_2$ & $f(x_2)$ & $f[x_1,x_2]$ & $f[x_0,x_1,x_2]$ & \\
    \hline
    $x_3$ & $f(x_3)$ & $f[x_3,x_2]$ & $f[x_2,x_1,x_3]$ & $f[x_3,x_2,x_1,x_0]$ \\
    \hline
\end{tabular}

NewTon 插值公式:

$
   N_n(x) = f(x_0)
    + f[x_0, x_1](x - x_0)
    + \cdots
    + f[x_0, \dots, x_n](x - x_0)\cdots(x - x_{n-1})
$

$
    R_{n}(x) = f[x,x_{0},\dots,x_{n}]\omega(x)
$


\subsection{特殊情况:因式分解法(推荐使用)}

\subsubsection*{模板 1:函数值相同}
若已知$f(a)=c,\quad f(b)=c\quad (a\ne b)$(两个点相同),则

\[
\boxed{H(x)=c+(x-a)(x-b)\,Q(x)}.
\]

如果题目知道 $H(x)$ 的次数上界为 $n$,则 $Q(x)$ 的次数$m \le n-2$。

$Q(x)=Ax^{m}+Bx^{m-1}+\cdots+Zx^{0}$

随后把剩余条件代入,解 $Q(x)$ 的系数即可。

扩展:如果3个点值相同则增加1项然后$Q(x)$次数变为$n-3$,以此类推。

\subsubsection*{模板 2:同一点函数值和一阶导都给出(重零点)}
若已知$H(a)=c,\quad H'(a)=c',$则令$G(x)=H(x)-c-c'(x-a),$设
\[
\boxed{H(x)=c+c'(x-a)+(x-a)^2\,Q(x)}.
\]
同样若 $\deg H\le n$,则 $\deg Q\le n-2$。

\subsection{Hermite插值余项}:

\[
R(x) = \frac{f^{(n+m)}(\xi) }{(n+m)!} \prod_{i=1}^n (x - x_i) \prod_{x_j \in D}(x-x_j)
\]

其中 $D$ 是 $m$ 个给出导数值的点的集合。

\section{最小二乘拟合\&最佳平方逼近}

\subsection{最小二乘拟合}

给定m+1个点x(下标从0开始)和y,用最小二乘法确定最优的直线$y=a+bx$。

(1)代入$x_i$和$y_i$,得到m+1超定方程组(格式抄出来即可):

\[
\left\{
    \begin{aligned}
        a+b x_0 &= y_0 \\
        a+b x_1 &= y_1 \\
        & \vdots \\
        a+b x_m &= y_m
    \end{aligned}
\right.
\Longleftrightarrow
\begin{bmatrix}
    1 & x_0 \\
    1 & x_1 \\
    \vdots & \vdots \\
    1 & x_m
\end{bmatrix}
\begin{bmatrix}
    a \\
    b
\end{bmatrix}
=
\begin{bmatrix}
    y_0 \\
    y_1 \\
    \vdots \\
    y_m
\end{bmatrix}
\Longrightarrow
\mathbf{A} \alpha = \beta
\]

(2)解法方程组(关键):

\[
A^T A \alpha = A^T \beta
\Longrightarrow
\begin{bmatrix}
    m+1 & \sum_{i=0}^m x_i \\
    \sum_{i=0}^m x_i & \sum_{i=0}^m x_i^2
\end{bmatrix}
\begin{bmatrix}
    a \\
    b
\end{bmatrix}
=
\begin{bmatrix}
    \sum_{i=0}^m y_i \\
    \sum_{i=0}^m y_i x_i
\end{bmatrix}
\]

拓展:对于n次多项式拟合,只需将矩阵$\mathbf{A}$的列数改为$n+1$,每一列对应$x^0,x^1,\dots,x^{n}$即可。$n$次的法方程组为:

\[
\begin{bmatrix}
&\sum_{i=0}^m 1 & \sum_{i=0}^m x_{i} &\dots & \sum_{i=0}^m x_{i}^n \\ 
&\sum_{i=0}^m x_{i} & \sum_{i=0}^m x_{i}^2 &\dots & \sum_{i=0}^m x_{i}^{n+1} \\
&\vdots &\vdots & \ddots & \vdots  \\
&\sum_{i=0}^m x_{i}^n & \sum_{i=0}^m x_{i}^{n+1} &\dots & \sum_{i=0}^m x_{i}^{2n} \\
\end{bmatrix}
\begin{bmatrix}
a_{0}  \\
a_{1} \\
\vdots \\
a_{n}
\end{bmatrix}
=
\begin{bmatrix}
\sum_{i=0}^m y_{i} \\
\sum_{i=0}^m y_{i}x_{i} \\ 
\vdots \\
\sum_{i=0}^m y_{i}x_{i}^n
\end{bmatrix}
\]

\subsection{最佳平方逼近}

求 $f(x)$在区间 $[a,b]$ 上的1次最佳平方逼近多项式。

(1)设置1次多项式为$\varphi(x) = a_0\varphi_0(x)+a_1 \varphi_1(x)$,其中$\varphi_0(x)=1$,$\varphi_1(x)=x$。

n次多项式则为 $\varphi(x) = a_0 \varphi_0(x) + a_1 \varphi_1(x)+\dots+a_n \varphi_n(x)$,$\varphi_i(x)=x^i$。

(2)写出法方程并求解:

\[
\begin{bmatrix}
    (\varphi_0,\varphi_0) & (\varphi_0,\varphi_1) \\
    (\varphi_1,\varphi_0) & (\varphi_1,\varphi_1)
\end{bmatrix}
\begin{bmatrix}
    a_0 \\
    a_1
\end{bmatrix}
=
\begin{bmatrix}
    (f,\varphi_0) \\
    (f,\varphi_1)
\end{bmatrix}
\]

其中 $(u,v) = \int_a^b u(x)v(x) \, dx$。

另外,n次平方逼近多项式的法方程:

\[
\begin{bmatrix}
&(\varphi_{0},\varphi_{0}) &\dots &(\varphi_{0},\varphi_{n-1}) \\
&\vdots & \ddots & \vdots  \\
&(\varphi_{n-1},\varphi_{0}) &\cdots & (\varphi_{n-1},\varphi_{n-1})
\end{bmatrix}
\begin{bmatrix}
a_{0}  \\
\vdots \\
a_{n-1}
\end{bmatrix}
=
\begin{bmatrix}
(\varphi_{0},f) \\
\vdots \\
(\varphi_{n-1},f)
\end{bmatrix}
\]

\section{数值积分}

\subsection{代数精度}

确定参数 $\{A_0,A_1,\dots,A_n\}$ 使得数值积分公式 $\int_{a}^{b} f(x) dx \approx \sum_{k=0}^n A_k f(x_k)$ 的代数精度尽可能高,指出代数精度是多少。

\begin{itemize}
    \item 取 $f(x)=1,x,x^2,\dots$ 求解方程组得到 $\{A_0,A_1,\dots,A_n\}$。
    \item 依次取 $f(x)=1,x,x^2,\dots$ ,验证求积公式是否成立,如果第一个不成立的是 $x^m$ ,则代数精度是 $m-1$。
\end{itemize}

\subsection{求积公式\&代数精度}

对定积分 $\int_a^b f(x) dx$ 分别使用辛普森公式、二等分的复化梯形公式和两个点的高斯公式进行计算并比较有效数字。

\subsubsection*{辛普森公式:}

\[
I_S = (b-a) \left[ \frac{1}{6} f(a) + \frac{4}{6} f(\frac{a+b}{2}) + \frac{1}{6} f(b) \right]
\]


\subsubsection*{二等分的复化梯形公式:}

$h=\frac{b-a}{2}$

\[
I_T = \frac{h}{2} \left[ f(a) + 2 f(\frac{a+b}{2}) + f(b) \right]
\]

\[
R_{T_2} = - \frac{(b-a)}{12} h^2 f''(\eta)
\]


\subsubsection*{2 个点的高斯公式:}

先将区间 $[a,b]$ 变换到 $[-1,1]$,令 $x = \frac{a+b}{2}+\frac{b-a}{2}t, dx = \frac{b-a}{2} dt$

\[
I_G =\int_a^b f(x) dx = \frac{b-a}{2} \int_{-1}^1 f\left( \frac{a+b}{2}+\frac{b-a}{2}t \right) dt
\]

令$\varphi(t) = f\left( \frac{a+b}{2}+\frac{b-a}{2}t \right)$,则

\[
\begin{aligned}
I_G=\int_{-1}^1f(t) dt 
&=\frac{b-a}{2} \left[ \varphi(-\frac{1}{\sqrt{3}}) + \frac{b-a}{2} \varphi(\frac{1}{\sqrt{3}}) \right] \\
&=\frac{b-a}{2} \left[ f\left( \frac{a+b}{2} - \frac{b-a}{2\sqrt{3}} \right) + f\left( \frac{a+b}{2} + \frac{b-a}{2\sqrt{3}} \right) \right]
\end{aligned}
\]

\subsubsection*{比较有效数字:}

首先根据不定积分公式求 $I=\int_a^b f(x) dx$ 的准确结果 $I^*$。

然后分别计算 $|I^*-I_S|$ 、 $|I^*-I_T|$和 $|I^*-I_G|$,记为 $e_S,e_T,e_G$。判断 $e_S,e_T,e_G$ 分别有多少位有效数字。

\textbf{有效数字}:$x=\pm 0.a_{1}a_{2}\dots a_{n} \times 10^m$,其中 m 是整数,$a_{1}$ 到 $a_{n}$ 都是 0 到 9 中的某个数,且 $a_{1} \neq 0$ 。如果有 $|x-x^*|<0.5 \times 10^{-k}$ ,则 $x^*$ 有 $m+k$ 位有效数字。

\subsection{求准确结果所需节点数}

对于 $I=\int_a^b f(x) dx$,如果要得到准确结果,牛顿-柯特斯公式和高斯公式分别需要多少节点数进行计算?

所需最低代数精度:$M = f(x)的最高次数$

对于n个节点的Gauss公式,代数精度为2n-1:$M \leq 2n-1$

对于n阶N-C公式:

当n为奇数时具有n次代数精度:$M \leq n$

当n为偶数时具有n+1次代数精度:$M \leq n+1$

\section{线性方程组}

已知线性方程组$\mathbf{A}x=b$

\subsection{LU分解}

对A进行LU分解(Doolittle分解),并利用分解结果求解方程组。

\[
\begin{bmatrix}
    a_{11} & a_{12} & \dots & a_{1n} \\
    a_{21} & a_{22} & \dots & a_{2n} \\
    \vdots & \vdots & \ddots & \vdots \\
    a_{n1} & a_{n2} & \dots & a_{nn} \\
\end{bmatrix}
= 
\begin{bmatrix}
    1 & \\
    l_{21} & 1 \\
    \vdots & \ddots & \ddots \\
    l_{n1} & l_{n2} & \dots & 1 \\
\end{bmatrix}
\begin{bmatrix}
    u_{11} & u_{12} & \dots & u_{1n} \\
    \ & u_{22} & \dots & u_{2n} \\
    \ & \ & \ddots & \vdots \\
    \ & \ & \ & u_{nn} \\
\end{bmatrix}
\]

\[
u_{kj} = a_{kj} -\sum_{q=1}^{k-1} l_{kq} u_{qi}  \qquad
l_{ik} = \frac{\left( a_{ik}-\sum_{q=1}^{k-1} l_{iq}u_{qk} \right)}{u_{kk}}
\]

技巧:卷面上直接抄上面的公式,然后使用紧凑法求解。

\begin{figure}[htbp]
\centering
\begin{tikzpicture}[
  every node/.style={},
]
\tikzset{stepline/.style={draw,thick}}

% --- 4x4 Matrix ---
\matrix (M) [
  matrix of math nodes,
  left delimiter={[}, right delimiter={]},
  row sep=6pt, column sep=10pt
] {
 u_{11} & u_{12} & u_{13} & u_{14} \\
 l_{21} & u_{22} & u_{23} & u_{24} \\
 l_{31} & l_{32} & u_{33} & u_{34} \\
 l_{41} & l_{42} & l_{43} & u_{44} \\
};

% --- L-shape "corner" lines (┌) for each step ---
% Step 1: corner at (2,2), spans to col 4 / row 4
\draw[stepline] (M-2-2.north west) -- (M-2-4.north east);
\draw[stepline] (M-2-2.north west) -- (M-4-2.south west);

% Step 2: corner at (3,3), spans to col 4 / row 4
\draw[stepline] (M-3-3.north west) -- (M-3-4.north east);
\draw[stepline] (M-3-3.north west) -- (M-4-3.south west);

% Step 3: corner at (4,4), just a small ┌
\draw[stepline] (M-4-4.north west) -- (M-4-4.north east);
\draw[stepline] (M-4-4.north west) -- (M-4-4.south west);

% --- Right-side labels ---
\node[right=6mm] at (M-1-4.east) {第1步};
\node[right=6mm] at (M-2-4.east) {第2步};
\node[right=6mm] at (M-3-4.east) {第3步};
\node[right=6mm] at (M-4-4.east) {第4步};


\end{tikzpicture}
\end{figure}


得到L和U之后求解 $Ly=b$,得到$y$。接着求解 $Ux=y$,得到 $x$

\subsection{Jacobi和Gauss-Seidel迭代}

给定初值 $x^{(0)}=\begin{bmatrix} 1 & 1 & 1\end{bmatrix}^T$出发,分别用Jacobi和Gauss-Seidel迭代1步,并且判断两个方法的收敛性。

\subsubsection*{Jacobi迭代}

将$Ax=b$改写为

\[
\left\{ 
    \begin{aligned}
    x_1^{(k+1)} &= \frac{1}{a_{11}} \left(- a_{12}x_2^{(k)} - a_{13}x_3^{(k)} ... + b_1\right) \\
    x_2^{(k+1)} &= \frac{1}{a_{22}} \left(- a_{21}x_1^{(k)} - a_{23}x_3^{(k)} ...  + b_2\right) \\
    \vdots \\
    x_n^{(k+1)} &= \frac{1}{a_{nn}} \left(- a_{n1}x_1^{(k)} - a_{n2}x_2^{(k)} ...  + b_n\right) \\
    \end{aligned}
    \right.
    \Longrightarrow
    \begin{aligned}
        &x^{(k+1)} = B_J x^{(k)} + g \\
        &B_J\text{从等式的右边抄}\\
        &\text{不存在的} x_m \text{项取0}\\
        &B_J=
            \begin{array}{c}
            \begin{array}{ccc}
            x_1 & x_2 & x_3
            \end{array}\\[2pt]
            \left[
            \begin{array}{ccc}
            \ & \ & \ \\
            \ & \ & \ \\
            \ & \ & \
            \end{array}
            \right]
            \end{array}
    \end{aligned}
\]

\subsubsection*{Gauss-Seidel}

参照Jacobi迭代,将$Ax=b$改写。只不过在计算$x_i^{(k+1)}$时,使用已经计算出的$x_j^{(k+1)}$,而不是$x_j^{(k)}$。口诀:“\textbf{右边的比左边的x的下标大就是k,比左边的x的下标小就是k+1}”

迭代格式: $X^{(k+1)} = B_G x^{(k)} + g$ 

求取$B_G$的时候需要将右边的 $k+1$ 项全部换为 $k$ 项。

\subsubsection*{收敛判断}

(1)\textbf{定理}(一般用不到):$Ax=b$,若$A$为严格对角占优矩阵,则$A$可逆,Jacobi和Gauss-Seidel迭代均收敛。(严格对角占优矩阵:$a_{ii} > \sum_{j=1,j\neq i}^n |a_{ij}|$)

(2)\textbf{定理(推荐使用)}:若$A$为对称正定矩阵,则GS迭代收敛;若$A$对称且对角元全正($a_{ii} > 0$)矩阵,则Jacobi方法收敛的\textbf{充要条件}为$A$和$2D-A$均为正定矩阵。 

正定判定方法:顺序主子式>0 。顺序主子式:对于$n$阶矩阵$A$,从左上角开始依次取$k\times k$子矩阵的行列式。

(3) \textbf{定理}:若$A$为不可约且弱对角占优,则称$A$为不可约对角占优,且Jacobi和GS迭代均收敛。

弱对角占优:$a_{ii} \geq \sum_{j=1,j\neq i}^n |a_{ij}|$

不可约矩阵:将$A$看作有向图,若这张图是强联通(任意两个节点互相可达)则称$A$为不可约矩阵,否则为可约矩阵。

(4) \textbf{谱半径判断法}(计算量过大,不到万不得已,不建议使用):
\begin{itemize}
    \item Jacobi迭代:$\rho(B_J) < 1$。\qquad $\rho(\cdot)$为矩阵特征值的绝对值的最大值。
    \item Gauss-Seidel迭代:$\rho(B_G) < 1$
\end{itemize}

\subsection{Recharson迭代}

令 $B=I-\alpha A$

迭代格式:$x^{(k+1)} = Bx^{(k)} + \alpha b$ 

收敛条件:$\lambda_B = 1-\alpha \lambda_A$,$\rho(B)<1 \Rightarrow |1-\alpha \lambda_A|<1$

$\lambda_{\min} \leq \lambda_A \leq \lambda_{\max} \Rightarrow \rho(B) = \max\{|1-\alpha \lambda_{\min}|, |1-\alpha \lambda_{\max}|\}$

最优参数 $\alpha$:$\alpha_{opt} = \frac{2}{\lambda_{\min} + \lambda_{\max}}$,$\rho(B)=\frac{\lambda_{\max}-\lambda_{\min}}{\lambda_{\max}+\lambda_{\min}} = \frac{\lambda_{\max}/\lambda_{\min}-1}{\lambda_{\max}/\lambda_{\min}+1}$

求$A$的特征值:$|A-\lambda I|=0$

\subsection{设计算法求矩阵第二大的特征值}

(1) 求特征值范围

\[
|\lambda_i -a_{ii}| \leq \sum_{j=1,j\neq i}^n |a_ij| \quad \text{依次对} i \text{行除了} a_{ii} \text{外的其他元素求和}
\]

(2) 位移,将原本第二模大特征值变为最大或最小特征值

$B=A-pI$,$B$的特征值变成$\lambda_i -p$

$B=A+pI$,$B$的特征值变成$\lambda_i +p$

(3) 求解 $B$ 的最大或最小特征值

取初始向量 $x^{(0)}\neq 0$,迭代$k=0,1,2,...$

幂法:求取B最大特征值

\[
\left\{
\begin{aligned}
    y^{(k)} &= \frac{x^{(k)}}{max(|x^{(k)}|)} \\
    x^{(k+1)} &= Bx^{(k)}
\end{aligned}
\right.
\qquad
\left\{
    \begin{aligned}
        \lim_{k \to \infty} y^{(k)} &= \frac{v_1}{\max(v_1)}  \; \text{特征向量}\\
        \lim_{k \to \infty} \max(x^{(k)}) &= \lambda_1 \; \text{最大特征值}
    \end{aligned}
\right.
\]

反幂法:求取B最小特征值

\[
\left\{
\begin{aligned}
    y^{(k)} &= \frac{x^{(k)}}{max(|x^{(k)}|)} \\
    Bx^{(k+1)} &= y^{(k)}
\end{aligned}
\right.
\qquad
\left\{
    \begin{aligned}
        \lim_{k \to \infty} y^{(k)} &= \frac{v_n}{\max(v_n)}  \; \text{特征向量}\\
        \lim_{k \to \infty} \max(x^{(k)}) &= \lambda_n \; \text{最小特征值}
    \end{aligned}
\right.
\]

(4) 将B的特征值转换回A的特征值




\section{非线性方程求根}

已知方程 $f(x) = 0$ ,在区间 $[a,b]$ 内有根,构造两个不同的收敛格式,证明收敛性并判断哪个格式收敛更快。

\subsection{迭代格式}

\subsubsection*{变换法}

通过变换(除法、开方等)将方程改写为 $x=\varphi(x)$ 

迭代格式: $x^{(k+1)} = \varphi(x^{(k)})$ 

全局收敛条件:1)在$[a,b]$内,$\varphi(x) \in [a,b]$;2)$\varphi$ 可导,且存在 $L<1$ 使得 $|\varphi'(x)|\leq L, \forall x \in [a,b]$

\subsubsection*{牛顿法}

\[
x_{k+1} = x_{k} - \frac{f(x)}{f'(x)}
\]

收敛条件:1)$f(a)f(b)<0$;2)$f'(x) \neq 0$,$\forall x \in [a,b]$;3)在$[a,b]$内,$f''(x)$ 不变号。则取初值 $x_0 \in [a,b]$ 且满足 $f(x_0)f''(x_0)>0$ ,则迭代收敛。

\subsection{收敛及收敛阶数}

设置$\alpha$为$f(\alpha)=0$在 $[a,b]$ 内的准确根(不用$x^*$是因为$\alpha$好写)。

局部收敛判定: 如果 $|\varphi'(\alpha)|<1$,且在 $\alpha$ 邻近连续,则初值够近就收敛。

收敛阶数:设 $x=\varphi(x)$的根具有连续$p$阶导数,且在根处 $\varphi'(\alpha)=\varphi''(\alpha)=...=\varphi^{(p-1)}(\alpha)=0$,$\varphi^{(p)}(\alpha) \neq 0$,则迭代格式的收敛阶数为 $p$ 。

% TODO: 重根的情况

\section{数值积分}

给定条件 $y' = f(x,y)$,初值 $y(1) = y_1$。(1)求解析解 (2)取步长为$h$,分别用隐式欧拉和改进欧拉方法求在$x=a$处的近似值。

\subsection{求解析解}

\subsubsection*{分离变量法(可分离变量时使用)}

如果 $f(x,y)$ 可以写成 $p(x)q(x)$ 的形式,则可分离变量:

$y'=f(x,y) \Rightarrow y'=\frac{dy}{dx}=p(x)q(x) \Rightarrow \frac{1}{p(y)}dy=q(x)dx$

$y = \int \frac{1}{p(y)}dy + C = \int q(x)dx +C$

\subsubsection*{公式法}

\textcircled{1} 改写为:$y' + p(x)y = q(x)$

\textcircled{2} 求 $\mu(x)$:$\mu(x) = e^{\int p(x) dx}$

技巧:$\int e^{(ax+b)} (cx+d) dx =  e^{ax+b}\left(\frac{c(ax-1)}{a^2}+\frac{d}{a}\right) + C$

\textcircled{3} 解$y(x)$:$\mu(x) y = \int \mu(x) q(x) dx + C \quad$

技巧:$y'(x) = ay+bx+c$ 输出 $y(x) = -\frac{b}{a^2} - \frac{bx}{a} - \frac{c}{a} + k_1 e^{ax}$


\subsection{欧拉法}

\subsubsection*{隐式欧拉} $y_{k+1} = y_k + h f(x_{k+1},y_{k+1})$

\subsubsection*{改进欧拉}

\[
\left\{
\begin{aligned}
y_{n+1}^{(p)} &= y_n + h\,f(x_n,y_n),\\
y_{n+1}^{(c)} &= y_n + \frac{h}{2}\left[f(x_n,y_n)+f\!\left(x_{n+1},y_{n+1}^{(p)}\right)\right].
\end{aligned}
\right.
\]

\section{附录:常用公式}

\begin{multicols}{2}

$(a+b)^3 = a^3 + 3a^2b + 3ab^2 + b^3$

$(a-b)^3 = a^3 - 3a^2b + 3ab^2 - b^3$

特别的:

$(a\pm b)^n = \sum_{k=0}^n C_n^k a^k (\pm b)^{n-k}$

$C_n^k = \frac{n!}{k!(n-k)!}$ \quad

\subsubsection*{积分公式}

分部积分法则:$\int u dv = uv - \int v du$

$\int x^n\,dx = \frac{x^{n+1}}{n+1}+C\ (n\neq -1)$

$\int \frac{1}{x}\,dx = \ln|x|+C$

$\int e^x\,dx = e^x + C$

$\int a^x\,dx = \frac{a^x}{\ln a}+C\ (a>0,a\neq 1)$

$\int \ln x\,dx = x\ln x - x + C\ (x>0)$

$\int \sin x\,dx = -\cos x + C$

$\int \cos x\,dx = \sin x + C$

$\int \frac{f'(x)}{f(x)}\,dx = \ln|f(x)| + C$

$\int e^{ax}\,dx = \frac{1}{a}e^{ax}+C\ (a\neq 0)$

$\int \sin(ax)\,dx = -\frac{1}{a}\cos(ax)+C$

$\int \cos(ax)\,dx = \frac{1}{a}\sin(ax)+C$


$\int \frac{1}{ax+b}\,dx = \frac{1}{a}\ln|ax+b|+C\ (a\neq 0)$
    
\end{multicols}




\end{document}
