\section{数值积分法}

\subsection{代数精度}

\begin{itemize}
    \item \textbf{定义}:定积分 $\int_{a}^b f(x) dx \approx \sum_{i=0}^n A_{i}f(x_{i})$,对于一切不高于 m 次的多项式都准确成立,但对 m+1 次不成立,则此公式的代数精度是 $m$
    \item \textbf{求解方法}:依次取 $f(x)=1,x,x^2,\dots$ ,验证求积公式是否成立,如果第一个不成立的是 $x^m$ ,则代数精度是 $m-1$
\end{itemize}


\subsection{单段的求积公式---Newton-Cotes求积公式(N-C公式)}

\[
\int_{a}^b f(x) dx \approx (b-a) \sum_{i=0}^n C_{i}^{(n)} f(x_{i})
\]

\begin{table}[htbp]
\centering
\renewcommand{\arraystretch}{1.2}
\setlength{\tabcolsep}{6pt}
\begin{tabular}{c|ccccccccc}
\hline
$n$ &
$c^{(n)}_0$ & $c^{(n)}_1$ & $c^{(n)}_2$ & $c^{(n)}_3$ & $c^{(n)}_4$ &
$c^{(n)}_5$ & $c^{(n)}_6$ & $c^{(n)}_7$ & $c^{(n)}_8$ \\
\hline
1 &
$\frac12$ & $\frac12$ & & & & & & & \\
2 &
$\frac16$ & $\frac23$ & $\frac16$ & & & & & & \\
3 &
$\frac18$ & $\frac38$ & $\frac38$ & $\frac18$ & & & & & \\
4 &
$\frac{7}{90}$ & $\frac{16}{45}$ & $\frac{2}{15}$ & $\frac{16}{45}$ & $\frac{7}{90}$ &
& & & \\
5 &
$\frac{19}{288}$ & $\frac{25}{96}$ & $\frac{25}{144}$ & $\frac{25}{144}$ & $\frac{25}{96}$ &
$\frac{19}{288}$ & & & \\
6 &
$\frac{41}{840}$ & $\frac{9}{35}$ & $\frac{9}{280}$ & $\frac{34}{105}$ & $\frac{9}{280}$ &
$\frac{9}{35}$ & $\frac{41}{840}$ & & \\
7 &
$\frac{751}{17280}$ & $\frac{3577}{17280}$ & $\frac{1323}{17280}$ & $\frac{2989}{17280}$ & $\frac{2989}{17280}$ &
$\frac{1323}{17280}$ & $\frac{3577}{17280}$ & $\frac{751}{17280}$ & \\
8 &
$\frac{989}{28350}$ & $\frac{5888}{28350}$ & $-\frac{928}{28350}$ & $\frac{10496}{28350}$ & $-\frac{4540}{28350}$ &
$\frac{10496}{28350}$ & $-\frac{928}{28350}$ & $\frac{5888}{28350}$ & $\frac{989}{28350}$ \\
\hline
\end{tabular}
\caption{Newton--Cotes(闭型)系数表 $c_k^{(n)}$(满足 $\sum_{k=0}^{n} c_k^{(n)}=1$,且 $c_k^{(n)}=c_{n-k}^{(n)}$)}
\end{table}

\begin{itemize}
    \item $n=1$,梯形:$T[f]=(b-a)\left[ \frac{1}{2} f(a) + \frac{1}{2} f(b) \right]$
    \item $n=2$,Simpson:$S[f]=(b-a)\left[ \frac{1}{6}f(a)+\frac{4}{6}f\left( \frac{b+a}{2} \right) +\frac{1}{6}f(b) \right]$
    \item $n=3$,Newton :$h=\frac{b-a}{3}$,$\int_{a}^b \approx \frac{b-a}{8}[f(a)+3f(a+h)+3f(a+2h)+f(b)]$
\end{itemize}

\subsection{复合求积公式}

\subsubsection{复合梯形}

$
T_{n}[f]= \frac{h}{2}\left( f(a)+ 2 \sum_{i=1}^{n-1} f(x_{i}) + f(b) \right) 
\qquad R_{T_{n}}(f) = - \frac{b-a}{12}h^2 f''(\eta)
$

\subsubsection{复合Simpson}

$
S_{n}[f] =
\frac{h}{3}
\Big[
f(x_0)+f(x_n)
+4\sum_{\text{奇 }i} f(x_i)
+2\sum_{\substack{\text{偶 }i\ i\neq 0,n}} f(x_i)
\Big]
$

$
R_{S_{n}}(f) = - \frac{h^5}{28880} h^4 f^{(4)}(\eta)
$

\subsubsection{变步长求积法}

计算 $T_{1},T_{2},\dots$,直到 $\frac{1}{3}|T_{2n}-T_{n}|\leq \varepsilon$

\subsection{Romberg求积法}

\[
\begin{matrix}
R_{0,0} \\
R_{1,0} & R_{1,1}\\
R_{2,0} & R_{2,1} & R_{2,2}\\
R_{3,0} & R_{3,1} & R_{3,2} & R_{3,3}\\
\vdots
\end{matrix}
\]

\begin{itemize}
    \item 第1列:
        \begin{itemize}
            \item 本质就是 k 次加密后的复合梯形公式:$R_{k,0}=T_{2^k}$
            \item 每次加密简便运算: $\displaystyle R_{k,0}=\frac12 R_{k-1,0}+h_k\sum f(a+(2i-1)h_k)$
        \end{itemize}
    \item 外推:
        \begin{itemize}
            \item 除了第一列之外的时候这个递推公式算
            \item $\displaystyle R_{k,j}=\frac{4^jR_{k,j-1}-R_{k-1,j-1}}{4^j-1}$
        \end{itemize}
    \item 停:$|R_{k,k}-R_{k,k-1}|<\varepsilon$
\end{itemize}

\subsection{Gauss求积公式}

在 $[-1,1]$ 上:$\int_{-1}^1f(t) dt = \sum_{k=1}^m A_{k} f(t_{k})$

区间变换:$令 x = \frac{a+b}{2}+\frac{b-a}{2}t, dx = \frac{b-a}{2} dt$


% Gauss--Legendre nodes x_k and weights A_k on [-1,1], n=1..5
\begin{table}[htbp]
\centering
\renewcommand{\arraystretch}{1.2}
\setlength{\tabcolsep}{12pt}
\begin{tabular}{|c|c|c|}
\hline
$n$ & $x_k$ & $A_k$ \\
\hline
1 & $0$ & $2$ \\
\hline
2 & $\pm 0.5773502692$ & $1$ \\
\hline
3 & $\pm 0.7745966692$ & $0.5555555556$ \\
  & $0$               & $0.8888888889$ \\
\hline
4 & $\pm 0.8611363116$ & $0.3478548451$ \\
  & $\pm 0.3399810436$ & $0.6521451549$ \\
\hline
5 & $\pm 0.9061798459$ & $0.2369268851$ \\
  & $\pm 0.5384693101$ & $0.4786286705$ \\
  & $0$                & $0.5688888889$ \\
\hline
\end{tabular}
\end{table}
