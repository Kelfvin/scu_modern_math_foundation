\section{最小二乘拟合\&最佳平方逼近}

\subsection{最小二乘拟合}

给定m+1个点x(下标从0开始)和y,用最小二乘法确定最优的直线$y=a+bx$。

(1)代入$x_i$和$y_i$,得到m+1超定方程组(格式抄出来即可):

\[
\left\{
    \begin{aligned}
        a+b x_0 &= y_0 \\
        a+b x_1 &= y_1 \\
        & \vdots \\
        a+b x_m &= y_m
    \end{aligned}
\right.
\Longleftrightarrow
\begin{bmatrix}
    1 & x_0 \\
    1 & x_1 \\
    \vdots & \vdots \\
    1 & x_m
\end{bmatrix}
\begin{bmatrix}
    a \\
    b
\end{bmatrix}
=
\begin{bmatrix}
    y_0 \\
    y_1 \\
    \vdots \\
    y_m
\end{bmatrix}
\Longrightarrow
\mathbf{A} \alpha = \beta
\]

(2)解法方程组(关键):

\[
A^T A \alpha = A^T \beta
\Longrightarrow
\begin{bmatrix}
    m+1 & \sum_{i=0}^m x_i \\
    \sum_{i=0}^m x_i & \sum_{i=0}^m x_i^2
\end{bmatrix}
\begin{bmatrix}
    a \\
    b
\end{bmatrix}
=
\begin{bmatrix}
    \sum_{i=0}^m y_i \\
    \sum_{i=0}^m y_i x_i
\end{bmatrix}
\]

拓展:对于n次多项式拟合,只需将矩阵$\mathbf{A}$的列数改为$n+1$,每一列对应$x^0,x^1,\dots,x^{n}$即可。$n$次的法方程组为:

\[
\begin{bmatrix}
&\sum_{i=0}^m 1 & \sum_{i=0}^m x_{i} &\dots & \sum_{i=0}^m x_{i}^n \\ 
&\sum_{i=0}^m x_{i} & \sum_{i=0}^m x_{i}^2 &\dots & \sum_{i=0}^m x_{i}^{n+1} \\
&\vdots &\vdots & \ddots & \vdots  \\
&\sum_{i=0}^m x_{i}^n & \sum_{i=0}^m x_{i}^{n+1} &\dots & \sum_{i=0}^m x_{i}^{2n} \\
\end{bmatrix}
\begin{bmatrix}
a_{0}  \\
a_{1} \\
\vdots \\
a_{n}
\end{bmatrix}
=
\begin{bmatrix}
\sum_{i=0}^m y_{i} \\
\sum_{i=0}^m y_{i}x_{i} \\ 
\vdots \\
\sum_{i=0}^m y_{i}x_{i}^n
\end{bmatrix}
\]

\subsection{最佳平方逼近}

求 $f(x)$在区间 $[a,b]$ 上的1次最佳平方逼近多项式。

(1)设置1次多项式为$\varphi(x) = a_0\varphi_0(x)+a_1 \varphi_1(x)$,其中$\varphi_0(x)=1$,$\varphi_1(x)=x$。

n次多项式则为 $\varphi(x) = a_0 \varphi_0(x) + a_1 \varphi_1(x)+\dots+a_n \varphi_n(x)$,$\varphi_i(x)=x^i$。

(2)写出法方程并求解:

\[
\begin{bmatrix}
    (\varphi_0,\varphi_0) & (\varphi_0,\varphi_1) \\
    (\varphi_1,\varphi_0) & (\varphi_1,\varphi_1)
\end{bmatrix}
\begin{bmatrix}
    a_0 \\
    a_1
\end{bmatrix}
=
\begin{bmatrix}
    (f,\varphi_0) \\
    (f,\varphi_1)
\end{bmatrix}
\]

其中 $(u,v) = \int_a^b u(x)v(x) \, dx$。

另外,n次平方逼近多项式的法方程:

\[
\begin{bmatrix}
&(\varphi_{0},\varphi_{0}) &\dots &(\varphi_{0},\varphi_{n-1}) \\
&\vdots & \ddots & \vdots  \\
&(\varphi_{n-1},\varphi_{0}) &\cdots & (\varphi_{n-1},\varphi_{n-1})
\end{bmatrix}
\begin{bmatrix}
a_{0}  \\
\vdots \\
a_{n-1}
\end{bmatrix}
=
\begin{bmatrix}
(\varphi_{0},f) \\
\vdots \\
(\varphi_{n-1},f)
\end{bmatrix}
\]