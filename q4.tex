\section{线性方程组}

已知线性方程组$\mathbf{A}x=b$

\subsection{LU分解}

对A进行LU分解(Doolittle分解),并利用分解结果求解方程组。

\[
\begin{bmatrix}
    a_{11} & a_{12} & \dots & a_{1n} \\
    a_{21} & a_{22} & \dots & a_{2n} \\
    \vdots & \vdots & \ddots & \vdots \\
    a_{n1} & a_{n2} & \dots & a_{nn} \\
\end{bmatrix}
= 
\begin{bmatrix}
    1 & \\
    l_{21} & 1 \\
    \vdots & \ddots & \ddots \\
    l_{n1} & l_{n2} & \dots & 1 \\
\end{bmatrix}
\begin{bmatrix}
    u_{11} & u_{12} & \dots & u_{1n} \\
    \ & u_{22} & \dots & u_{2n} \\
    \ & \ & \ddots & \vdots \\
    \ & \ & \ & u_{nn} \\
\end{bmatrix}
\]

\[
u_{kj} = a_{kj} -\sum_{q=1}^{k-1} l_{kq} u_{qi}  \qquad
l_{ik} = \frac{\left( a_{ik}-\sum_{q=1}^{k-1} l_{iq}u_{qk} \right)}{u_{kk}}
\]

技巧:卷面上直接抄上面的公式,然后使用紧凑法求解。

\begin{figure}[htbp]
\centering
\begin{tikzpicture}[
  every node/.style={},
]
\tikzset{stepline/.style={draw,thick}}

% --- 4x4 Matrix ---
\matrix (M) [
  matrix of math nodes,
  left delimiter={[}, right delimiter={]},
  row sep=6pt, column sep=10pt
] {
 u_{11} & u_{12} & u_{13} & u_{14} \\
 l_{21} & u_{22} & u_{23} & u_{24} \\
 l_{31} & l_{32} & u_{33} & u_{34} \\
 l_{41} & l_{42} & l_{43} & u_{44} \\
};

% --- L-shape "corner" lines (┌) for each step ---
% Step 1: corner at (2,2), spans to col 4 / row 4
\draw[stepline] (M-2-2.north west) -- (M-2-4.north east);
\draw[stepline] (M-2-2.north west) -- (M-4-2.south west);

% Step 2: corner at (3,3), spans to col 4 / row 4
\draw[stepline] (M-3-3.north west) -- (M-3-4.north east);
\draw[stepline] (M-3-3.north west) -- (M-4-3.south west);

% Step 3: corner at (4,4), just a small ┌
\draw[stepline] (M-4-4.north west) -- (M-4-4.north east);
\draw[stepline] (M-4-4.north west) -- (M-4-4.south west);

% --- Right-side labels ---
\node[right=6mm] at (M-1-4.east) {第1步};
\node[right=6mm] at (M-2-4.east) {第2步};
\node[right=6mm] at (M-3-4.east) {第3步};
\node[right=6mm] at (M-4-4.east) {第4步};


\end{tikzpicture}
\end{figure}


得到L和U之后求解 $Ly=b$,得到$y$。接着求解 $Ux=y$,得到 $x$

\subsection{Jacobi和Gauss-Seidel迭代}

给定初值 $x^{(0)}=\begin{bmatrix} 1 & 1 & 1\end{bmatrix}^T$出发,分别用Jacobi和Gauss-Seidel迭代1步,并且判断两个方法的收敛性。

\subsubsection*{Jacobi迭代}

将$Ax=b$改写为

\[
\left\{ 
    \begin{aligned}
    x_1^{(k+1)} &= \frac{1}{a_{11}} \left(- a_{12}x_2^{(k)} - a_{13}x_3^{(k)} ... + b_1\right) \\
    x_2^{(k+1)} &= \frac{1}{a_{22}} \left(- a_{21}x_1^{(k)} - a_{23}x_3^{(k)} ...  + b_2\right) \\
    \vdots \\
    x_n^{(k+1)} &= \frac{1}{a_{nn}} \left(- a_{n1}x_1^{(k)} - a_{n2}x_2^{(k)} ...  + b_n\right) \\
    \end{aligned}
    \right.
    \Longrightarrow
    \begin{aligned}
        &x^{(k+1)} = B_J x^{(k)} + g \\
        &B_J\text{从等式的右边抄}\\
        &\text{不存在的} x_m \text{项取0}\\
        &B_J=
            \begin{array}{c}
            \begin{array}{ccc}
            x_1 & x_2 & x_3
            \end{array}\\[2pt]
            \left[
            \begin{array}{ccc}
            \ & \ & \ \\
            \ & \ & \ \\
            \ & \ & \
            \end{array}
            \right]
            \end{array}
    \end{aligned}
\]

\subsubsection*{Gauss-Seidel}

参照Jacobi迭代,将$Ax=b$改写。只不过在计算$x_i^{(k+1)}$时,使用已经计算出的$x_j^{(k+1)}$,而不是$x_j^{(k)}$。口诀:“\textbf{右边的比左边的x的下标大就是k,比左边的x的下标小就是k+1}”

迭代格式: $X^{(k+1)} = B_G x^{(k)} + g$ 

求取$B_G$的时候需要将右边的 $k+1$ 项全部换为 $k$ 项。

\subsubsection*{收敛判断}

(1)\textbf{定理}(一般用不到):$Ax=b$,若$A$为严格对角占优矩阵,则$A$可逆,Jacobi和Gauss-Seidel迭代均收敛。(严格对角占优矩阵:$a_{ii} > \sum_{j=1,j\neq i}^n |a_{ij}|$)

(2)\textbf{定理(推荐使用)}:若$A$为对称正定矩阵,则GS迭代收敛;若$A$对称且对角元全正($a_{ii} > 0$)矩阵,则Jacobi方法收敛的\textbf{充要条件}为$A$和$2D-A$均为正定矩阵。 

正定判定方法:顺序主子式>0 。顺序主子式:对于$n$阶矩阵$A$,从左上角开始依次取$k\times k$子矩阵的行列式。

(3) \textbf{定理}:若$A$为不可约且弱对角占优,则称$A$为不可约对角占优,且Jacobi和GS迭代均收敛。

弱对角占优:$a_{ii} \geq \sum_{j=1,j\neq i}^n |a_{ij}|$

不可约矩阵:将$A$看作有向图,若这张图是强联通(任意两个节点互相可达)则称$A$为不可约矩阵,否则为可约矩阵。

(4) \textbf{谱半径判断法}(计算量过大,不到万不得已,不建议使用):
\begin{itemize}
    \item Jacobi迭代:$\rho(B_J) < 1$。\qquad $\rho(\cdot)$为矩阵特征值的绝对值的最大值。
    \item Gauss-Seidel迭代:$\rho(B_G) < 1$
\end{itemize}

\subsection{Recharson迭代}

令 $B=I-\alpha A$

迭代格式:$x^{(k+1)} = Bx^{(k)} + \alpha b$ 

收敛条件:$\lambda_B = 1-\alpha \lambda_A$,$\rho(B)<1 \Rightarrow |1-\alpha \lambda_A|<1$

$\lambda_{\min} \leq \lambda_A \leq \lambda_{\max} \Rightarrow \rho(B) = \max\{|1-\alpha \lambda_{\min}|, |1-\alpha \lambda_{\max}|\}$

最优参数 $\alpha$:$\alpha_{opt} = \frac{2}{\lambda_{\min} + \lambda_{\max}}$,$\rho(B)=\frac{\lambda_{\max}-\lambda_{\min}}{\lambda_{\max}+\lambda_{\min}} = \frac{\lambda_{\max}/\lambda_{\min}-1}{\lambda_{\max}/\lambda_{\min}+1}$

求$A$的特征值:$|A-\lambda I|=0$

\subsection{设计算法求矩阵第二大的特征值}

(1) 求特征值范围

\[
|\lambda_i -a_{ii}| \leq \sum_{j=1,j\neq i}^n |a_ij| \quad \text{依次对} i \text{行除了} a_{ii} \text{外的其他元素求和}
\]

(2) 位移,将原本第二模大特征值变为最大或最小特征值

$B=A-pI$,$B$的特征值变成$\lambda_i -p$

$B=A+pI$,$B$的特征值变成$\lambda_i +p$

(3) 求解 $B$ 的最大或最小特征值

取初始向量 $x^{(0)}\neq 0$,迭代$k=0,1,2,...$

幂法:求取B最大特征值

\[
\left\{
\begin{aligned}
    y^{(k)} &= \frac{x^{(k)}}{max(|x^{(k)}|)} \\
    x^{(k+1)} &= Bx^{(k)}
\end{aligned}
\right.
\qquad
\left\{
    \begin{aligned}
        \lim_{k \to \infty} y^{(k)} &= \frac{v_1}{\max(v_1)}  \; \text{特征向量}\\
        \lim_{k \to \infty} \max(x^{(k)}) &= \lambda_1 \; \text{最大特征值}
    \end{aligned}
\right.
\]

反幂法:求取B最小特征值

\[
\left\{
\begin{aligned}
    y^{(k)} &= \frac{x^{(k)}}{max(|x^{(k)}|)} \\
    Bx^{(k+1)} &= y^{(k)}
\end{aligned}
\right.
\qquad
\left\{
    \begin{aligned}
        \lim_{k \to \infty} y^{(k)} &= \frac{v_n}{\max(v_n)}  \; \text{特征向量}\\
        \lim_{k \to \infty} \max(x^{(k)}) &= \lambda_n \; \text{最小特征值}
    \end{aligned}
\right.
\]

(4) 将B的特征值转换回A的特征值


